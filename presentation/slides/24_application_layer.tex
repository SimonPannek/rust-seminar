% !TeX root = ../main.tex
\begin{frame}{Application Layer communication}
    \center{
        \begin{minipage}{4cm}
            \begin{table}
                \begin{tabular}{l}
                    \toprule
                    \textbf{OSI layers}           \\
                    \midrule
                    \begin{overprint}
                        \onslide<1>
                        Application Layer
                        \onslide<2>
                        \textbf{Application Layer}
                    \end{overprint} \\
                    Application Layer             \\
                    Presentation Layer            \\
                    Session Layer                 \\
                    \begin{overprint}
                        \onslide<1>
                        \textbf{Transport Layer}
                        \onslide<2>
                        Transport Layer
                    \end{overprint} \\
                    Network Layer                 \\
                    Data Link Layer               \\
                    Physical Layer                \\
                    \bottomrule
                \end{tabular}
            \end{table}
        \end{minipage}
    }

    \enote{
        \item Layers (just in case):
        \begin{enumerate}
            \item Application Layer: Network Process to Application
            \item Presentation Layer: Data representation and Encryption
            \item Session Layer: Interhost communication
            \item Transport Layer: End-to-End connections
            \item Network Layer: Path Determination and logical addressing
            \item Data Link Layer: Physical addressing
            \item Physical Layer: Media, signal and binary transmission
        \end{enumerate}
        \item Showcase how Tokio can be used together with higher level protocols
        \item Hypertext Transfer Protocol (HTTP) for data transmission for the World Wide Web (usually based on TCP)
    }
\end{frame}
