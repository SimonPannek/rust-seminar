% !TeX root = ../main.tex
\begin{frame}{Networking in Rust}
    \Huge
    \centerline{Ownership, borrowing and lifetimes}

    \pause

    \br

    \Large
    \centerline{\so No dangling references, illegal memory access and memory leaks}

    \pause

    \br

    \small
    \centerline{(Except if you are explicitely working with \rust{unsafe} code)}

    \enote{
        \item Ownership, borrowing and lifetimes play an important role in networking code
        \item Ownership:
        \begin{enumerate}
            \item A part of the code can own a piece of memory
            \item Function is called with an owned value \so the piece of memory gets moved into that function
        \end{enumerate}
        \item Borrowing:
        \begin{enumerate}
            \item No complete access is needed \so a reference can be borrowed
            \item There are also mutable reference: Only one is allowed at the same time
            \item Useful whne working with network applications (prevents the occurence of data races if a thread tires
                  to read or write to a buffer while another thread is already writing to it)
            \item We are not going to take a look at \rust{std::sync::Arc}
        \end{enumerate}
        \item Lifetimes:
        \begin{enumerate}
            \item Lifetime determined by the code block it was created \so variable gets out of scope \so the
                  corresponding piece of memory is freed
            \item If a value gets moved into another function, its lifetime changes with it
        \end{enumerate}
    }
\end{frame}
