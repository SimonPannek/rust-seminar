% !TeX root = ../main.tex
% Add the above to each chapter to make compiling the PDF easier in some editors.

\section{Application layer communication}
In chapter 2 and 3, transport layer protocols were used to explicitly send text messages formatted as byte slices from
a client to a server. This may be enough for simple applications like a chat server, but in many cases another
abstraction on top of that is needed to send special data like structs over a connection. This is where protocols
belonging to the application layer come into place.

The following chapter will give a small insight into the principle how the asynchronous communication using tokio can
also be used together with higher level protocols. In particular the Hypertext Transfer Protocol (HTTP), a protocol
used for data transmission for the World Wide Web is going to be utilized in order to send a \rust{struct} from a
client to the server and receive a response as a confirmation. HTTP is usually based on the TCP protocol but can also
be implemented using other transport protocols, which offer a confirmation, if sending data was successful.

\subsection{Serialization}
Before sending anything over HTTP, first a piece of data to actually send is needed. For this we are going to introduce
the \rust{struct} called \rust{Data} in figure 10, which will both be implemented in the HTTP client and server.

\begin{figure}[ht]
    \begin{minted}{Rust}
#[derive(Serialize, Deserialize, Debug)]
struct Data {
    number: u32,
    boolean: bool,
}
    \end{minted}
    \caption{Data struct}
\end{figure}

Even though we do not want to use raw TCP communication to send our packages, HTTP still is a text based protocol. This
means before being able to send the struct from the client to the server, we have to transform it from a \rust{struct}
to a \rust{String} and back, a principle called serialization and deserialization.

This is why we are going to use a crate called serde. Serde allows us to let our \rust{struct} derive \rust{Serialize}
and \rust{Deserialize}. This way the compiler is capable of providing a basic implementation to convert \rust{Data} to
a JSON \rust{String} and the other way around. JSON stands for JavaScript Object Notation and is just a standard file
format typically used for exactly this purpose.

The additional deriving of \rust{Debug} implements the trait \rust{std::fmt::Debug}. This serves the purpose of
printing the struct in a human readable format, typically used for debugging purposes.

\subsection{HTTP communication}
Now that it is possible to convert the struct into a format which is sendable over HTTP, next a server is needed to
accept incoming requests. There are many crates, which enable you to accept and handle HTTP requests in Rust. For our
HTTP server implementation, we are going to use a crate called warp.

Warp is simple web server framework, which builds on top of the crate hyper, a HTTP implementation for Rust
\cite{warp-doc}. Warp makes use of asynchronous programming and seamlessly works together with tokio. An important
concept of warp are the so called filters: \rust{warp::Filter} is a struct, which extracts data from a request, handles
the data and sends a response back to the sender of the request. The \rust{Filter} implements two important methods:

First the method \rust{and}, which takes another \rust{Filter} as an argument and returns
them as a new, combined \rust{Filter}. This way you can chain together multiple \rust{Filter} instances and filter
requests more precisely.

Secondly the method \rust{map}, which takes a function as an argument. This function receives the extracted data from
the request and can map it to another value. This value in return gets sent back to the client sending the request.

\rust{Filter} also implements more methods useful for rejecting certain requests or customizing filters to take
different sets of arguments in a request, but those methods will not be needed for our example.

In figure 11 you can find a small example implementation of a web server written using warp.

\begin{figure}[ht]
    \begin{minted}{Rust}
let filter = warp::'\textcolor{macro}{path!}'("data")
    .and(warp::post())
    .and(warp::body::json())
    .map(|data: Data|*format!("Received: {:?}", data));

warp::serve(filter).run(ADDRESS).await;
    \end{minted}
    \caption{Warp HTTP server}
\end{figure}

First a \rust{Filter}, which matches a path of a possible request sent to the server, has to be created. Here the macro
\nohighlight{warp::path!} can be used which, returns a \rust{Filter} matching the exact path segment. For our example
we are going to send the data to the path "/data".

Next we have to specify, how the request exactly looks like. This enables us to automatically reject invalid requests.
In this case, we only want to use a POST request, a method supported by HTTP and used to accept data sent in the body
of the incoming request. We can do this by creating a \rust{Filter} requiring the POST method using \rust{warp::post()}
and then combining both filters using the previously mentioned \rust{and} method.

When implementing the \rust{struct} which actually should get sent, we used serialization into the JSON format. This is
why the web server should expect a body encoded in JSON and try to parse the data into a \rust{Data} instance. A
\rust{Filter} which does just that can be created using the function \rust{warp::body::json()} and then again be
combined using \rust{and}.

Finally the extracted \rust{Data} instance can be mapped to a response, which can be send back to the client. Here a
simple formatting to a \rust{String} using the \nohighlight{format!} macro works fine. The \rust{{:?}} brackets the the
formatting macro to use the \rust{std::fmt::Debug} trait derived by \rust{Data}.

The function \rust{warp::serve} takes the \rust{Filter} defined above as an argument and creates a new instance of the
struct \rust{warp::Server}. This server represents a web server filtering requests according to the filter given at
initialization.

To run this server, the struct implements the asynchronous method \rust{run}, taking the address it should listen to
as an argument and running the \rust{Server} on the current thread until the program is closed. \rust{ADDRESS} here has
the type \rust{([u8, 4], u16)}, a tupel with the ip address as a byte slice with a length of 4 and the port represented
as an unsigned short.

To send the struct to this web server, we need to create a HTTP client. Reqwest is a crate often used for sending
simple HTTP requests to a web server and similar to warp also makes use of asynchronous methods and requires some kind
of runtime like tokio, at least when using its default API. In figure 12 there is an example implementation of a
reqwest HTTP client sending a custom struct of the type \rust{Data} to the warp server.

\begin{figure}[ht]
    \begin{minted}{Rust}
let client = reqwest::Client::new();

let response = client
    .post(URL)
    .json(&Data {
        number: 5,
        boolean: true,
    })
    .send()
    .await
    .expect("Sending request");

println!("{}", response.text()*.await.expect("Get text"));
    \end{minted}
    \caption{Reqwest HTTP client}
\end{figure}

In \textcolor{orange}{line 1} an instance of the type \rust{reqwest::Client} gets created in order to send asynchronous
requests to the server. On this client we can call the method \rust{post}, which creates a
\rust{reqwest::RequestBuilder} representing a post request. The argument \rust{URL} is equal to the HTTP url of the
targeted web server. To give it the data as a payload, the method \rust{json} is called with a reference to our custom
\rust{Data} object. This method uses the \rust{Serialize} trait of the struct to convert the data into a JSON
\rust{String}. Then the \rust{send} method can be invoked to actually construct the request and send it to the target
URL. To save the result of the request, the returned value is first unwrapped using \rust{await} and \rust{expect}.

\rust{response} has the type \rust{reqwest::Response}. To print the resulting text answered by the server, the
asynchronous method \rust{text} can be used. The \rust{String} has to be unwrapped using \rust{await} and
\rust{expect}, similar to the returned value of \rust{send}.

If we run the example of figure 12 with the web server of figure 11 running at the same time, the client prints the
response \nohighlight{Received:} \nohighlight{Data { number: 5, boolean: true }}.
