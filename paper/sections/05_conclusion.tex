% !TeX root = ../main.tex
% Add the above to each chapter to make compiling the PDF easier in some editors.

\section{Conclusion}
One final question which might come up is why Rust is not used more frequently for network applications. In the
introduction, it was mentioned that most network APIs are written in C or C++. The Rust Survey of 2020 suggests, that a
huge issue is the adoption of the language in companies \cite{rust-survey}. This makes sense as switching the language
of an existing code base requires a lot of refactoring and introducing a new language adds the overhead of the
employees first having to learn this new language. Another common reason is the steep learning curve related to the
rather strict concepts of ownership and lifetimes the compiler enforces.

In the course of this paper, we have taken a look at general problems when working with network applications in other
programming languages and the approach of using Rust to work around these boundaries. The standard library of Rust was
introduced and it was demonstrated how it can be utilized to establish a connection between a client and a server using
the Transmission Control Protocol (TCP). After that, a User Datagram Protocol (UDP) client and server implementation
was shown, which allowed multiple clients to message the receiver at the same time. Next, we took a look at
asynchronous programming and how the introduction of a custom scheduler can replace blocking operations by swapping out
tasks, which have reached an \rust{await} point and have to wait for another task to evaluate. In this context, the
synchronous TCP implementation from earlier was converted into an asynchronous one, and the advantage of not having to
create a thread for each connection, that should be maintained at the same time, was demonstrated. In the end, a
\rust{struct} was sent from a client to a server using the Hypertext Transfer Protocol (HTTP) in order to show how
asynchronous programming can also be used in network applications on the application layer.

Many aspects were not considered in the previous chapters. To improve scalability one might consider to use a
decentralized approach featuring load balancing with multiple servers handling connections and possibly communicating
with each other in the background. Also a lot of possible security vulnerabilities were not considered yet: Rust is
able to avoid security issues caused by null pointers, dangling pointers or data races, but these are not the only
types of errors that can occur. For instance, if a server is accepting input from other clients, it is usually best to
parse and verify the input received before handling the message instead of blindly trusting the incoming data.

In conclusion, Rust is a very powerful language for writing programs to run on network servers. It is often important
to deploy a secure and efficient language in the backend of an application as the server underlying a service can be
considered the bottleneck most of the time when it comes to verifying and handling user requests. Slowing down or even
outages of such applications might cause a lot of damage and should be avoided at any cost. Security vulnerabilities
could also cause interruption of service or even accidental disclosure of sensitive user data, depending on which type
of application is affected.

Here Rust's checks at compile time come to shine, because many of the common security issues based on memory unsafety
can be prevented in advance. In addition, shifting most of these checks into the process of compiling the code, Rust
does not check for things like ownership and borrowing at runtime, thus making the executable even faster while still
retaining all safety features.
